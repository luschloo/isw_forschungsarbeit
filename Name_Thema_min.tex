\documentclass[ a4paper,
                oneside,
                toc=bibliography,
                toc=listof
                ]{scrbook}

\usepackage[ngerman]{babel} % If the thesis is in English
%\usepackage[english, ngerman]{babel} % If the thesis is in German


% This class does the ISW styling for you (together with scrbook).
%
% It handles the following:
% - Proper input and font encoding (Just type, don't care about the LaTeX compiler you use or how to type German umlauts)
% - Fonts with ligatures and kerning (Tex Gyre fonts are used, part of every LaTeX installation, text is nice to read)
% - Bibliography styling for biblatex (declare your bibliography file and you are ready to go)
% - Provide command for title page (\makeISWtitle) and declaration of originality ( \declarationOfOriginality)
% - Loads packages "biblatex" and "graphics"
\usepackage[
    type=study, % master, bachelor, bachelorproject
]{iswthesis}

%Path to .bib-File(s) for BibLatex
\addbibresource{bibliography.bib}
% \addbibresource{someOtherBibFile}

\author{Lukas Schlotter}
\placeOfBirth{Stuttgart}
\major{Mechatronik}
\title{jbjhkbkj}
\titleTranslated{Wie man einen Hamster trainiert}
\matrnr{3668915}
\date{\today}
\supervisor{My supervisor, M.Sc.}
\professor{Prof. Dr.-Ing. Oliver Riedel}

\begin{document} 
    \frontmatter
    \makeISWtitle
    
    \cleardoublepage
	\setcounter{page}{1} % start at page (i) after title page
    \declarationOfOriginality

    % Kurzfassung/Abstract
    
    \cleardoublepage
    \tableofcontents
    

    \mainmatter
    
    \chapter{Einleitung}
    Warum startet das hier mit ner 0?
    \section{Motivation}
    \begin{figure}[h]
    	\centering
    	\includegraphics[width=1.0\linewidth]{./images/Test}
    	\caption{BPMN Prozess bei einem Taxiruf}
    	\label{fig:bpmn prozess}
    \end{figure}
    Dieses Bild zeigt blabla bla von dem Buch \cite{Tantau2013} und auch \cite{Kohm2013}
    
    \begin{table}[h!]
    	\centering
    	\begin{tabular}{||c c c c||} 
    		\hline
    		Col1 & Col2 & Col2 & Col3 \\ [0.5ex] 
    		\hline\hline
    		1 & 6 & 87837 & 787 \\ 
    		2 & 7 & 78 & 5415 \\
    		3 & 545 & 778 & 7507 \\
    		4 & 545 & 18744 & 7560 \\
    		5 & 88 & 788 & 6344 \\ [1ex] 
    		\hline
    	\end{tabular}
    	\caption{Table to test captions and labels.}
    	\label{table:1}
    \end{table}

   
    \backmatter
    \cleardoublepage
    \printbibliography

    \cleardoublepage
    \listoffigures
    \cleardoublepage
    \listoftables
    \cleardoublepage
    % Acronyms

    % Appendix, if needed:


\end{document}